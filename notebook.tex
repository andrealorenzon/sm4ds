
% Default to the notebook output style

    


% Inherit from the specified cell style.




    
\documentclass[11pt]{article}

    
    
    \usepackage[T1]{fontenc}
    % Nicer default font (+ math font) than Computer Modern for most use cases
    \usepackage{mathpazo}

    % Basic figure setup, for now with no caption control since it's done
    % automatically by Pandoc (which extracts ![](path) syntax from Markdown).
    \usepackage{graphicx}
    % We will generate all images so they have a width \maxwidth. This means
    % that they will get their normal width if they fit onto the page, but
    % are scaled down if they would overflow the margins.
    \makeatletter
    \def\maxwidth{\ifdim\Gin@nat@width>\linewidth\linewidth
    \else\Gin@nat@width\fi}
    \makeatother
    \let\Oldincludegraphics\includegraphics
    % Set max figure width to be 80% of text width, for now hardcoded.
    \renewcommand{\includegraphics}[1]{\Oldincludegraphics[width=.8\maxwidth]{#1}}
    % Ensure that by default, figures have no caption (until we provide a
    % proper Figure object with a Caption API and a way to capture that
    % in the conversion process - todo).
    \usepackage{caption}
    \DeclareCaptionLabelFormat{nolabel}{}
    \captionsetup{labelformat=nolabel}

    \usepackage{adjustbox} % Used to constrain images to a maximum size 
    \usepackage{xcolor} % Allow colors to be defined
    \usepackage{enumerate} % Needed for markdown enumerations to work
    \usepackage{geometry} % Used to adjust the document margins
    \usepackage{amsmath} % Equations
    \usepackage{amssymb} % Equations
    \usepackage{textcomp} % defines textquotesingle
    % Hack from http://tex.stackexchange.com/a/47451/13684:
    \AtBeginDocument{%
        \def\PYZsq{\textquotesingle}% Upright quotes in Pygmentized code
    }
    \usepackage{upquote} % Upright quotes for verbatim code
    \usepackage{eurosym} % defines \euro
    \usepackage[mathletters]{ucs} % Extended unicode (utf-8) support
    \usepackage[utf8x]{inputenc} % Allow utf-8 characters in the tex document
    \usepackage{fancyvrb} % verbatim replacement that allows latex
    \usepackage{grffile} % extends the file name processing of package graphics 
                         % to support a larger range 
    % The hyperref package gives us a pdf with properly built
    % internal navigation ('pdf bookmarks' for the table of contents,
    % internal cross-reference links, web links for URLs, etc.)
    \usepackage{hyperref}
    \usepackage{longtable} % longtable support required by pandoc >1.10
    \usepackage{booktabs}  % table support for pandoc > 1.12.2
    \usepackage[inline]{enumitem} % IRkernel/repr support (it uses the enumerate* environment)
    \usepackage[normalem]{ulem} % ulem is needed to support strikethroughs (\sout)
                                % normalem makes italics be italics, not underlines
    

    
    
    % Colors for the hyperref package
    \definecolor{urlcolor}{rgb}{0,.145,.698}
    \definecolor{linkcolor}{rgb}{.71,0.21,0.01}
    \definecolor{citecolor}{rgb}{.12,.54,.11}

    % ANSI colors
    \definecolor{ansi-black}{HTML}{3E424D}
    \definecolor{ansi-black-intense}{HTML}{282C36}
    \definecolor{ansi-red}{HTML}{E75C58}
    \definecolor{ansi-red-intense}{HTML}{B22B31}
    \definecolor{ansi-green}{HTML}{00A250}
    \definecolor{ansi-green-intense}{HTML}{007427}
    \definecolor{ansi-yellow}{HTML}{DDB62B}
    \definecolor{ansi-yellow-intense}{HTML}{B27D12}
    \definecolor{ansi-blue}{HTML}{208FFB}
    \definecolor{ansi-blue-intense}{HTML}{0065CA}
    \definecolor{ansi-magenta}{HTML}{D160C4}
    \definecolor{ansi-magenta-intense}{HTML}{A03196}
    \definecolor{ansi-cyan}{HTML}{60C6C8}
    \definecolor{ansi-cyan-intense}{HTML}{258F8F}
    \definecolor{ansi-white}{HTML}{C5C1B4}
    \definecolor{ansi-white-intense}{HTML}{A1A6B2}

    % commands and environments needed by pandoc snippets
    % extracted from the output of `pandoc -s`
    \providecommand{\tightlist}{%
      \setlength{\itemsep}{0pt}\setlength{\parskip}{0pt}}
    \DefineVerbatimEnvironment{Highlighting}{Verbatim}{commandchars=\\\{\}}
    % Add ',fontsize=\small' for more characters per line
    \newenvironment{Shaded}{}{}
    \newcommand{\KeywordTok}[1]{\textcolor[rgb]{0.00,0.44,0.13}{\textbf{{#1}}}}
    \newcommand{\DataTypeTok}[1]{\textcolor[rgb]{0.56,0.13,0.00}{{#1}}}
    \newcommand{\DecValTok}[1]{\textcolor[rgb]{0.25,0.63,0.44}{{#1}}}
    \newcommand{\BaseNTok}[1]{\textcolor[rgb]{0.25,0.63,0.44}{{#1}}}
    \newcommand{\FloatTok}[1]{\textcolor[rgb]{0.25,0.63,0.44}{{#1}}}
    \newcommand{\CharTok}[1]{\textcolor[rgb]{0.25,0.44,0.63}{{#1}}}
    \newcommand{\StringTok}[1]{\textcolor[rgb]{0.25,0.44,0.63}{{#1}}}
    \newcommand{\CommentTok}[1]{\textcolor[rgb]{0.38,0.63,0.69}{\textit{{#1}}}}
    \newcommand{\OtherTok}[1]{\textcolor[rgb]{0.00,0.44,0.13}{{#1}}}
    \newcommand{\AlertTok}[1]{\textcolor[rgb]{1.00,0.00,0.00}{\textbf{{#1}}}}
    \newcommand{\FunctionTok}[1]{\textcolor[rgb]{0.02,0.16,0.49}{{#1}}}
    \newcommand{\RegionMarkerTok}[1]{{#1}}
    \newcommand{\ErrorTok}[1]{\textcolor[rgb]{1.00,0.00,0.00}{\textbf{{#1}}}}
    \newcommand{\NormalTok}[1]{{#1}}
    
    % Additional commands for more recent versions of Pandoc
    \newcommand{\ConstantTok}[1]{\textcolor[rgb]{0.53,0.00,0.00}{{#1}}}
    \newcommand{\SpecialCharTok}[1]{\textcolor[rgb]{0.25,0.44,0.63}{{#1}}}
    \newcommand{\VerbatimStringTok}[1]{\textcolor[rgb]{0.25,0.44,0.63}{{#1}}}
    \newcommand{\SpecialStringTok}[1]{\textcolor[rgb]{0.73,0.40,0.53}{{#1}}}
    \newcommand{\ImportTok}[1]{{#1}}
    \newcommand{\DocumentationTok}[1]{\textcolor[rgb]{0.73,0.13,0.13}{\textit{{#1}}}}
    \newcommand{\AnnotationTok}[1]{\textcolor[rgb]{0.38,0.63,0.69}{\textbf{\textit{{#1}}}}}
    \newcommand{\CommentVarTok}[1]{\textcolor[rgb]{0.38,0.63,0.69}{\textbf{\textit{{#1}}}}}
    \newcommand{\VariableTok}[1]{\textcolor[rgb]{0.10,0.09,0.49}{{#1}}}
    \newcommand{\ControlFlowTok}[1]{\textcolor[rgb]{0.00,0.44,0.13}{\textbf{{#1}}}}
    \newcommand{\OperatorTok}[1]{\textcolor[rgb]{0.40,0.40,0.40}{{#1}}}
    \newcommand{\BuiltInTok}[1]{{#1}}
    \newcommand{\ExtensionTok}[1]{{#1}}
    \newcommand{\PreprocessorTok}[1]{\textcolor[rgb]{0.74,0.48,0.00}{{#1}}}
    \newcommand{\AttributeTok}[1]{\textcolor[rgb]{0.49,0.56,0.16}{{#1}}}
    \newcommand{\InformationTok}[1]{\textcolor[rgb]{0.38,0.63,0.69}{\textbf{\textit{{#1}}}}}
    \newcommand{\WarningTok}[1]{\textcolor[rgb]{0.38,0.63,0.69}{\textbf{\textit{{#1}}}}}
    
    
    % Define a nice break command that doesn't care if a line doesn't already
    % exist.
    \def\br{\hspace*{\fill} \\* }
    % Math Jax compatability definitions
    \def\gt{>}
    \def\lt{<}
    % Document parameters
    \title{Assignment4}
    
    
    

    % Pygments definitions
    
\makeatletter
\def\PY@reset{\let\PY@it=\relax \let\PY@bf=\relax%
    \let\PY@ul=\relax \let\PY@tc=\relax%
    \let\PY@bc=\relax \let\PY@ff=\relax}
\def\PY@tok#1{\csname PY@tok@#1\endcsname}
\def\PY@toks#1+{\ifx\relax#1\empty\else%
    \PY@tok{#1}\expandafter\PY@toks\fi}
\def\PY@do#1{\PY@bc{\PY@tc{\PY@ul{%
    \PY@it{\PY@bf{\PY@ff{#1}}}}}}}
\def\PY#1#2{\PY@reset\PY@toks#1+\relax+\PY@do{#2}}

\expandafter\def\csname PY@tok@w\endcsname{\def\PY@tc##1{\textcolor[rgb]{0.73,0.73,0.73}{##1}}}
\expandafter\def\csname PY@tok@c\endcsname{\let\PY@it=\textit\def\PY@tc##1{\textcolor[rgb]{0.25,0.50,0.50}{##1}}}
\expandafter\def\csname PY@tok@cp\endcsname{\def\PY@tc##1{\textcolor[rgb]{0.74,0.48,0.00}{##1}}}
\expandafter\def\csname PY@tok@k\endcsname{\let\PY@bf=\textbf\def\PY@tc##1{\textcolor[rgb]{0.00,0.50,0.00}{##1}}}
\expandafter\def\csname PY@tok@kp\endcsname{\def\PY@tc##1{\textcolor[rgb]{0.00,0.50,0.00}{##1}}}
\expandafter\def\csname PY@tok@kt\endcsname{\def\PY@tc##1{\textcolor[rgb]{0.69,0.00,0.25}{##1}}}
\expandafter\def\csname PY@tok@o\endcsname{\def\PY@tc##1{\textcolor[rgb]{0.40,0.40,0.40}{##1}}}
\expandafter\def\csname PY@tok@ow\endcsname{\let\PY@bf=\textbf\def\PY@tc##1{\textcolor[rgb]{0.67,0.13,1.00}{##1}}}
\expandafter\def\csname PY@tok@nb\endcsname{\def\PY@tc##1{\textcolor[rgb]{0.00,0.50,0.00}{##1}}}
\expandafter\def\csname PY@tok@nf\endcsname{\def\PY@tc##1{\textcolor[rgb]{0.00,0.00,1.00}{##1}}}
\expandafter\def\csname PY@tok@nc\endcsname{\let\PY@bf=\textbf\def\PY@tc##1{\textcolor[rgb]{0.00,0.00,1.00}{##1}}}
\expandafter\def\csname PY@tok@nn\endcsname{\let\PY@bf=\textbf\def\PY@tc##1{\textcolor[rgb]{0.00,0.00,1.00}{##1}}}
\expandafter\def\csname PY@tok@ne\endcsname{\let\PY@bf=\textbf\def\PY@tc##1{\textcolor[rgb]{0.82,0.25,0.23}{##1}}}
\expandafter\def\csname PY@tok@nv\endcsname{\def\PY@tc##1{\textcolor[rgb]{0.10,0.09,0.49}{##1}}}
\expandafter\def\csname PY@tok@no\endcsname{\def\PY@tc##1{\textcolor[rgb]{0.53,0.00,0.00}{##1}}}
\expandafter\def\csname PY@tok@nl\endcsname{\def\PY@tc##1{\textcolor[rgb]{0.63,0.63,0.00}{##1}}}
\expandafter\def\csname PY@tok@ni\endcsname{\let\PY@bf=\textbf\def\PY@tc##1{\textcolor[rgb]{0.60,0.60,0.60}{##1}}}
\expandafter\def\csname PY@tok@na\endcsname{\def\PY@tc##1{\textcolor[rgb]{0.49,0.56,0.16}{##1}}}
\expandafter\def\csname PY@tok@nt\endcsname{\let\PY@bf=\textbf\def\PY@tc##1{\textcolor[rgb]{0.00,0.50,0.00}{##1}}}
\expandafter\def\csname PY@tok@nd\endcsname{\def\PY@tc##1{\textcolor[rgb]{0.67,0.13,1.00}{##1}}}
\expandafter\def\csname PY@tok@s\endcsname{\def\PY@tc##1{\textcolor[rgb]{0.73,0.13,0.13}{##1}}}
\expandafter\def\csname PY@tok@sd\endcsname{\let\PY@it=\textit\def\PY@tc##1{\textcolor[rgb]{0.73,0.13,0.13}{##1}}}
\expandafter\def\csname PY@tok@si\endcsname{\let\PY@bf=\textbf\def\PY@tc##1{\textcolor[rgb]{0.73,0.40,0.53}{##1}}}
\expandafter\def\csname PY@tok@se\endcsname{\let\PY@bf=\textbf\def\PY@tc##1{\textcolor[rgb]{0.73,0.40,0.13}{##1}}}
\expandafter\def\csname PY@tok@sr\endcsname{\def\PY@tc##1{\textcolor[rgb]{0.73,0.40,0.53}{##1}}}
\expandafter\def\csname PY@tok@ss\endcsname{\def\PY@tc##1{\textcolor[rgb]{0.10,0.09,0.49}{##1}}}
\expandafter\def\csname PY@tok@sx\endcsname{\def\PY@tc##1{\textcolor[rgb]{0.00,0.50,0.00}{##1}}}
\expandafter\def\csname PY@tok@m\endcsname{\def\PY@tc##1{\textcolor[rgb]{0.40,0.40,0.40}{##1}}}
\expandafter\def\csname PY@tok@gh\endcsname{\let\PY@bf=\textbf\def\PY@tc##1{\textcolor[rgb]{0.00,0.00,0.50}{##1}}}
\expandafter\def\csname PY@tok@gu\endcsname{\let\PY@bf=\textbf\def\PY@tc##1{\textcolor[rgb]{0.50,0.00,0.50}{##1}}}
\expandafter\def\csname PY@tok@gd\endcsname{\def\PY@tc##1{\textcolor[rgb]{0.63,0.00,0.00}{##1}}}
\expandafter\def\csname PY@tok@gi\endcsname{\def\PY@tc##1{\textcolor[rgb]{0.00,0.63,0.00}{##1}}}
\expandafter\def\csname PY@tok@gr\endcsname{\def\PY@tc##1{\textcolor[rgb]{1.00,0.00,0.00}{##1}}}
\expandafter\def\csname PY@tok@ge\endcsname{\let\PY@it=\textit}
\expandafter\def\csname PY@tok@gs\endcsname{\let\PY@bf=\textbf}
\expandafter\def\csname PY@tok@gp\endcsname{\let\PY@bf=\textbf\def\PY@tc##1{\textcolor[rgb]{0.00,0.00,0.50}{##1}}}
\expandafter\def\csname PY@tok@go\endcsname{\def\PY@tc##1{\textcolor[rgb]{0.53,0.53,0.53}{##1}}}
\expandafter\def\csname PY@tok@gt\endcsname{\def\PY@tc##1{\textcolor[rgb]{0.00,0.27,0.87}{##1}}}
\expandafter\def\csname PY@tok@err\endcsname{\def\PY@bc##1{\setlength{\fboxsep}{0pt}\fcolorbox[rgb]{1.00,0.00,0.00}{1,1,1}{\strut ##1}}}
\expandafter\def\csname PY@tok@kc\endcsname{\let\PY@bf=\textbf\def\PY@tc##1{\textcolor[rgb]{0.00,0.50,0.00}{##1}}}
\expandafter\def\csname PY@tok@kd\endcsname{\let\PY@bf=\textbf\def\PY@tc##1{\textcolor[rgb]{0.00,0.50,0.00}{##1}}}
\expandafter\def\csname PY@tok@kn\endcsname{\let\PY@bf=\textbf\def\PY@tc##1{\textcolor[rgb]{0.00,0.50,0.00}{##1}}}
\expandafter\def\csname PY@tok@kr\endcsname{\let\PY@bf=\textbf\def\PY@tc##1{\textcolor[rgb]{0.00,0.50,0.00}{##1}}}
\expandafter\def\csname PY@tok@bp\endcsname{\def\PY@tc##1{\textcolor[rgb]{0.00,0.50,0.00}{##1}}}
\expandafter\def\csname PY@tok@fm\endcsname{\def\PY@tc##1{\textcolor[rgb]{0.00,0.00,1.00}{##1}}}
\expandafter\def\csname PY@tok@vc\endcsname{\def\PY@tc##1{\textcolor[rgb]{0.10,0.09,0.49}{##1}}}
\expandafter\def\csname PY@tok@vg\endcsname{\def\PY@tc##1{\textcolor[rgb]{0.10,0.09,0.49}{##1}}}
\expandafter\def\csname PY@tok@vi\endcsname{\def\PY@tc##1{\textcolor[rgb]{0.10,0.09,0.49}{##1}}}
\expandafter\def\csname PY@tok@vm\endcsname{\def\PY@tc##1{\textcolor[rgb]{0.10,0.09,0.49}{##1}}}
\expandafter\def\csname PY@tok@sa\endcsname{\def\PY@tc##1{\textcolor[rgb]{0.73,0.13,0.13}{##1}}}
\expandafter\def\csname PY@tok@sb\endcsname{\def\PY@tc##1{\textcolor[rgb]{0.73,0.13,0.13}{##1}}}
\expandafter\def\csname PY@tok@sc\endcsname{\def\PY@tc##1{\textcolor[rgb]{0.73,0.13,0.13}{##1}}}
\expandafter\def\csname PY@tok@dl\endcsname{\def\PY@tc##1{\textcolor[rgb]{0.73,0.13,0.13}{##1}}}
\expandafter\def\csname PY@tok@s2\endcsname{\def\PY@tc##1{\textcolor[rgb]{0.73,0.13,0.13}{##1}}}
\expandafter\def\csname PY@tok@sh\endcsname{\def\PY@tc##1{\textcolor[rgb]{0.73,0.13,0.13}{##1}}}
\expandafter\def\csname PY@tok@s1\endcsname{\def\PY@tc##1{\textcolor[rgb]{0.73,0.13,0.13}{##1}}}
\expandafter\def\csname PY@tok@mb\endcsname{\def\PY@tc##1{\textcolor[rgb]{0.40,0.40,0.40}{##1}}}
\expandafter\def\csname PY@tok@mf\endcsname{\def\PY@tc##1{\textcolor[rgb]{0.40,0.40,0.40}{##1}}}
\expandafter\def\csname PY@tok@mh\endcsname{\def\PY@tc##1{\textcolor[rgb]{0.40,0.40,0.40}{##1}}}
\expandafter\def\csname PY@tok@mi\endcsname{\def\PY@tc##1{\textcolor[rgb]{0.40,0.40,0.40}{##1}}}
\expandafter\def\csname PY@tok@il\endcsname{\def\PY@tc##1{\textcolor[rgb]{0.40,0.40,0.40}{##1}}}
\expandafter\def\csname PY@tok@mo\endcsname{\def\PY@tc##1{\textcolor[rgb]{0.40,0.40,0.40}{##1}}}
\expandafter\def\csname PY@tok@ch\endcsname{\let\PY@it=\textit\def\PY@tc##1{\textcolor[rgb]{0.25,0.50,0.50}{##1}}}
\expandafter\def\csname PY@tok@cm\endcsname{\let\PY@it=\textit\def\PY@tc##1{\textcolor[rgb]{0.25,0.50,0.50}{##1}}}
\expandafter\def\csname PY@tok@cpf\endcsname{\let\PY@it=\textit\def\PY@tc##1{\textcolor[rgb]{0.25,0.50,0.50}{##1}}}
\expandafter\def\csname PY@tok@c1\endcsname{\let\PY@it=\textit\def\PY@tc##1{\textcolor[rgb]{0.25,0.50,0.50}{##1}}}
\expandafter\def\csname PY@tok@cs\endcsname{\let\PY@it=\textit\def\PY@tc##1{\textcolor[rgb]{0.25,0.50,0.50}{##1}}}

\def\PYZbs{\char`\\}
\def\PYZus{\char`\_}
\def\PYZob{\char`\{}
\def\PYZcb{\char`\}}
\def\PYZca{\char`\^}
\def\PYZam{\char`\&}
\def\PYZlt{\char`\<}
\def\PYZgt{\char`\>}
\def\PYZsh{\char`\#}
\def\PYZpc{\char`\%}
\def\PYZdl{\char`\$}
\def\PYZhy{\char`\-}
\def\PYZsq{\char`\'}
\def\PYZdq{\char`\"}
\def\PYZti{\char`\~}
% for compatibility with earlier versions
\def\PYZat{@}
\def\PYZlb{[}
\def\PYZrb{]}
\makeatother


    % Exact colors from NB
    \definecolor{incolor}{rgb}{0.0, 0.0, 0.5}
    \definecolor{outcolor}{rgb}{0.545, 0.0, 0.0}



    
    % Prevent overflowing lines due to hard-to-break entities
    \sloppy 
    % Setup hyperref package
    \hypersetup{
      breaklinks=true,  % so long urls are correctly broken across lines
      colorlinks=true,
      urlcolor=urlcolor,
      linkcolor=linkcolor,
      citecolor=citecolor,
      }
    % Slightly bigger margins than the latex defaults
    
    \geometry{verbose,tmargin=1in,bmargin=1in,lmargin=1in,rmargin=1in}
    
    

    \begin{document}
    
    
    \maketitle
    
    

    
    \begin{Verbatim}[commandchars=\\\{\}]
{\color{incolor}In [{\color{incolor}63}]:} \PY{k+kn}{library}\PY{p}{(}DAAG\PY{p}{)}
         \PY{k+kn}{library}\PY{p}{(}psych\PY{p}{)}  \PY{c+c1}{\PYZsh{}pairplot}
\end{Verbatim}


    \begin{Verbatim}[commandchars=\\\{\}]

Attaching package: ‘psych’

The following object is masked from ‘package:DAAG’:

    cities


    \end{Verbatim}

    \begin{Verbatim}[commandchars=\\\{\}]
{\color{incolor}In [{\color{incolor}65}]:} pairs.panels\PY{p}{(}nihills\PY{p}{,} gap\PY{o}{=}\PY{l+m}{0}\PY{p}{)}
\end{Verbatim}


    \begin{center}
    \adjustimage{max size={0.9\linewidth}{0.9\paperheight}}{output_1_0.png}
    \end{center}
    { \hspace*{\fill} \\}
    
    The following investigates the consequences of not using a logarithmic
transformation for the nihills data analysis. The second differs from
the first in having a dist × climb interaction term, additional to
linear terms in dist and climb.

\begin{enumerate}
\def\labelenumi{(\alph{enumi})}
\tightlist
\item
  Fit the two models:
\end{enumerate}

\texttt{nihills.lm\ \textless{}-\ lm(time\ ˜\ dist+climb,\ data=nihills)\ nihills2.lm\ \textless{}-\ lm(time\ ˜\ dist+climb+dist:climb,\ data=nihills)\ anova(nihills.lm,\ nihills2.lm)}

    \begin{Verbatim}[commandchars=\\\{\}]
{\color{incolor}In [{\color{incolor}62}]:} \PY{k+kp}{summary}\PY{p}{(}nihills\PY{p}{)}
\end{Verbatim}


    
    \begin{verbatim}
      dist            climb           time            timef       
 Min.   : 2.500   Min.   : 750   Min.   :0.3247   Min.   :0.4092  
 1st Qu.: 4.000   1st Qu.:1205   1st Qu.:0.4692   1st Qu.:0.6158  
 Median : 4.500   Median :1500   Median :0.5506   Median :0.7017  
 Mean   : 5.778   Mean   :2098   Mean   :0.8358   Mean   :1.1107  
 3rd Qu.: 5.800   3rd Qu.:2245   3rd Qu.:0.7857   3rd Qu.:1.0014  
 Max.   :18.900   Max.   :8775   Max.   :3.9028   Max.   :5.9856  
    \end{verbatim}

    
    \begin{Verbatim}[commandchars=\\\{\}]
{\color{incolor}In [{\color{incolor}28}]:} nihills.lm  \PY{o}{\PYZlt{}\PYZhy{}} lm\PY{p}{(}time \PY{o}{\PYZti{}} dist\PY{o}{+}climb           \PY{p}{,} data\PY{o}{=}nihills\PY{p}{)}
         nihills2.lm \PY{o}{\PYZlt{}\PYZhy{}} lm\PY{p}{(}time \PY{o}{\PYZti{}} dist\PY{o}{+}climb\PY{o}{+}dist\PY{o}{:}climb\PY{p}{,} data\PY{o}{=}nihills\PY{p}{)}
\end{Verbatim}


    \begin{Verbatim}[commandchars=\\\{\}]
{\color{incolor}In [{\color{incolor}29}]:} \PY{k+kp}{summary}\PY{p}{(}nihills.lm\PY{p}{)}
\end{Verbatim}


    
    \begin{verbatim}

Call:
lm(formula = time ~ dist + climb, data = nihills)

Residuals:
     Min       1Q   Median       3Q      Max 
-0.19857 -0.04824  0.01701  0.05539  0.21083 

Coefficients:
              Estimate Std. Error t value Pr(>|t|)    
(Intercept) -2.286e-01  4.025e-02  -5.679 1.47e-05 ***
dist         1.008e-01  1.382e-02   7.293 4.72e-07 ***
climb        2.298e-04  2.893e-05   7.941 1.31e-07 ***
---
Signif. codes:  0 ‘***’ 0.001 ‘**’ 0.01 ‘*’ 0.05 ‘.’ 0.1 ‘ ’ 1

Residual standard error: 0.0973 on 20 degrees of freedom
Multiple R-squared:  0.9852,	Adjusted R-squared:  0.9838 
F-statistic: 667.6 on 2 and 20 DF,  p-value: < 2.2e-16

    \end{verbatim}

    
    \begin{Verbatim}[commandchars=\\\{\}]
{\color{incolor}In [{\color{incolor}34}]:} par\PY{p}{(}mfrow\PY{o}{=}\PY{k+kt}{c}\PY{p}{(}\PY{l+m}{2}\PY{p}{,}\PY{l+m}{2}\PY{p}{)}\PY{p}{)}
         plot\PY{p}{(}nihills.lm\PY{p}{)}
\end{Verbatim}


    \begin{center}
    \adjustimage{max size={0.9\linewidth}{0.9\paperheight}}{output_6_0.png}
    \end{center}
    { \hspace*{\fill} \\}
    
    \begin{Verbatim}[commandchars=\\\{\}]
{\color{incolor}In [{\color{incolor}30}]:} \PY{k+kp}{summary}\PY{p}{(}nihills2.lm\PY{p}{)}
\end{Verbatim}


    
    \begin{verbatim}

Call:
lm(formula = time ~ dist + climb + dist:climb, data = nihills)

Residuals:
     Min       1Q   Median       3Q      Max 
-0.07854 -0.03182 -0.01334  0.02894  0.08711 

Coefficients:
             Estimate Std. Error t value Pr(>|t|)    
(Intercept) 4.677e-02  3.744e-02   1.249   0.2267    
dist        6.962e-02  7.427e-03   9.374 1.48e-08 ***
climb       9.988e-05  2.040e-05   4.896   0.0001 ***
dist:climb  9.964e-06  1.171e-06   8.509 6.62e-08 ***
---
Signif. codes:  0 ‘***’ 0.001 ‘**’ 0.01 ‘*’ 0.05 ‘.’ 0.1 ‘ ’ 1

Residual standard error: 0.04552 on 19 degrees of freedom
Multiple R-squared:  0.9969,	Adjusted R-squared:  0.9964 
F-statistic:  2058 on 3 and 19 DF,  p-value: < 2.2e-16

    \end{verbatim}

    
    \begin{Verbatim}[commandchars=\\\{\}]
{\color{incolor}In [{\color{incolor}36}]:} par\PY{p}{(}mfrow\PY{o}{=}\PY{k+kt}{c}\PY{p}{(}\PY{l+m}{2}\PY{p}{,}\PY{l+m}{2}\PY{p}{)}\PY{p}{)}
         plot\PY{p}{(}nihills2.lm\PY{p}{)}
\end{Verbatim}


    \begin{Verbatim}[commandchars=\\\{\}]
Warning message in sqrt(crit * p * (1 - hh)/hh):
“NaNs produced”Warning message in sqrt(crit * p * (1 - hh)/hh):
“NaNs produced”
    \end{Verbatim}

    \begin{center}
    \adjustimage{max size={0.9\linewidth}{0.9\paperheight}}{output_8_1.png}
    \end{center}
    { \hspace*{\fill} \\}
    
    \begin{Verbatim}[commandchars=\\\{\}]
{\color{incolor}In [{\color{incolor}32}]:} anova\PY{p}{(}nihills.lm\PY{p}{,} nihills2.lm\PY{p}{)}
\end{Verbatim}


    \begin{tabular}{r|llllll}
 Res.Df & RSS & Df & Sum of Sq & F & Pr(>F)\\
\hline
	 20           & 0.18936094   & NA           &        NA    &      NA      &           NA\\
	 19           & 0.03936124   &  1           & 0.1499997    & 72.4061      & 6.623143e-08\\
\end{tabular}


    
    \textbf{Commentary:}

Including outsider values, the model

    \subsection{part b}\label{part-b}

\begin{enumerate}
\def\labelenumi{(\alph{enumi})}
\setcounter{enumi}{1}
\tightlist
\item
  Using the F-test result, make a tentative choice of model, and proceed
  to examine diagnostic plots. Are there any problematic observations?
  What happens if these points are removed? Refit both of the above
  models, and check the diagnostics again.
\end{enumerate}

    \begin{Verbatim}[commandchars=\\\{\}]
{\color{incolor}In [{\color{incolor}79}]:} outsiders \PY{o}{=} nihills\PY{p}{[}\PY{k+kt}{c}\PY{p}{(}\PY{l+m}{10}\PY{p}{,}\PY{l+m}{17}\PY{p}{,}\PY{l+m}{19}\PY{p}{)}\PY{p}{,}\PY{p}{]}
         \PY{k+kp}{print}\PY{p}{(}\PY{l+s}{\PYZdq{}}\PY{l+s}{Outsiders:\PYZdq{}}\PY{p}{)}
         outsiders
         nihills.clean \PY{o}{=} nihills\PY{p}{[}\PY{o}{\PYZhy{}}\PY{k+kt}{c}\PY{p}{(}\PY{l+m}{10}\PY{p}{,}\PY{l+m}{17}\PY{p}{,}\PY{l+m}{19}\PY{p}{)}\PY{p}{,}\PY{p}{]}
         \PY{k+kp}{print}\PY{p}{(}\PY{l+s}{\PYZdq{}}\PY{l+s}{Dataset \PYZhy{} outsiders:\PYZdq{}}\PY{p}{)}
         nihills.clean
         pairs.panels\PY{p}{(}nihills.clean\PY{p}{,} gap\PY{o}{=}\PY{l+m}{0}\PY{p}{)}
\end{Verbatim}


    \begin{Verbatim}[commandchars=\\\{\}]
[1] "Outsiders:"

    \end{Verbatim}

    \begin{tabular}{r|llll}
  & dist & climb & time & timef\\
\hline
	Annalong Horseshoe & 12.0     & 5080     & 1.949167 & 2.480556\\
	Flagstaff to Carling & 11.0     & 3000     & 1.456944 & 2.034444\\
	Seven Sevens & 18.9     & 8775     & 3.902778 & 5.985556\\
\end{tabular}


    
    \begin{Verbatim}[commandchars=\\\{\}]
[1] "Dataset - outsiders:"

    \end{Verbatim}

    \begin{tabular}{r|llll}
  & dist & climb & time & timef\\
\hline
	Binevenagh & 7.5       & 1740      & 0.8583333 & 1.0644444\\
	Slieve Gullion & 4.2       & 1110      & 0.4666667 & 0.6230556\\
	Glenariff Mountain & 5.9       & 1210      & 0.7030556 & 0.8869444\\
	Donard \& Commedagh & 6.8       & 3300      & 1.0386111 & 1.2141667\\
	McVeigh Classic & 5.0       & 1200      & 0.5411111 & 0.6375000\\
	Tollymore Mountain & 4.8       &  950      & 0.4833333 & 0.5886111\\
	Slieve Martin & 4.3       & 1600      & 0.5505556 & 0.7016667\\
	Moughanmore & 3.0       & 1500      & 0.4636111 & 0.6475000\\
	Hen \& Cock & 2.5       & 1500      & 0.4497222 & 0.6075000\\
	Monument Race & 4.0       & 1000      & 0.4716667 & 0.5947222\\
	Loughshannagh Horseshoe & 4.3       & 1700      & 0.6469444 & 0.8822222\\
	Rocky & 4.0       & 1300      & 0.5230556 & 0.6652778\\
	Meelbeg Meelmore & 3.5       & 1800      & 0.4544444 & 0.6086111\\
	Donard Forest & 4.5       & 1400      & 0.5186111 & 0.6433333\\
	Slieve Donard & 5.5       & 2790      & 0.9483333 & 1.2086111\\
	Slieve Bearnagh & 4.0       & 2690      & 0.6877778 & 0.7991667\\
	Lurig Challenge & 4.0       & 1000      & 0.4347222 & 0.5755556\\
	Scrabo Hill Race & 2.9       &  750      & 0.3247222 & 0.4091667\\
	Slieve Gallion & 4.6       & 1440      & 0.6361111 & 0.7494444\\
	BARF Turkey Trot & 5.7       & 1430      & 0.7130556 & 0.9383333\\
\end{tabular}


    
    \begin{center}
    \adjustimage{max size={0.9\linewidth}{0.9\paperheight}}{output_12_4.png}
    \end{center}
    { \hspace*{\fill} \\}
    
    \begin{Verbatim}[commandchars=\\\{\}]
{\color{incolor}In [{\color{incolor}80}]:} nihills.clean.lm  \PY{o}{\PYZlt{}\PYZhy{}} lm\PY{p}{(}time \PY{o}{\PYZti{}} dist\PY{o}{+}climb           \PY{p}{,} data\PY{o}{=}nihills.clean\PY{p}{)}
         nihills2.clean.lm \PY{o}{\PYZlt{}\PYZhy{}} lm\PY{p}{(}time \PY{o}{\PYZti{}} dist\PY{o}{+}climb\PY{o}{+}dist\PY{o}{:}climb\PY{p}{,} data\PY{o}{=}nihills.clean\PY{p}{)}
\end{Verbatim}


    \begin{Verbatim}[commandchars=\\\{\}]
{\color{incolor}In [{\color{incolor}81}]:} \PY{k+kp}{summary}\PY{p}{(}nihills.clean.lm\PY{p}{)}
\end{Verbatim}


    
    \begin{verbatim}

Call:
lm(formula = time ~ dist + climb, data = nihills.clean)

Residuals:
      Min        1Q    Median        3Q       Max 
-0.085066 -0.030909 -0.008891  0.040014  0.075740 

Coefficients:
              Estimate Std. Error t value Pr(>|t|)    
(Intercept) -5.407e-02  4.040e-02  -1.338    0.198    
dist         8.815e-02  9.009e-03   9.785 2.13e-08 ***
climb        1.584e-04  1.726e-05   9.177 5.37e-08 ***
---
Signif. codes:  0 ‘***’ 0.001 ‘**’ 0.01 ‘*’ 0.05 ‘.’ 0.1 ‘ ’ 1

Residual standard error: 0.04524 on 17 degrees of freedom
Multiple R-squared:  0.9462,	Adjusted R-squared:  0.9399 
F-statistic: 149.4 on 2 and 17 DF,  p-value: 1.632e-11

    \end{verbatim}

    
    \begin{Verbatim}[commandchars=\\\{\}]
{\color{incolor}In [{\color{incolor}82}]:} par\PY{p}{(}mfrow\PY{o}{=}\PY{k+kt}{c}\PY{p}{(}\PY{l+m}{2}\PY{p}{,}\PY{l+m}{2}\PY{p}{)}\PY{p}{)}
         plot\PY{p}{(}nihills.clean.lm\PY{p}{)}
\end{Verbatim}


    \begin{center}
    \adjustimage{max size={0.9\linewidth}{0.9\paperheight}}{output_15_0.png}
    \end{center}
    { \hspace*{\fill} \\}
    
    \begin{Verbatim}[commandchars=\\\{\}]
{\color{incolor}In [{\color{incolor}83}]:} \PY{k+kp}{summary}\PY{p}{(}nihills2.clean.lm\PY{p}{)}
\end{Verbatim}


    
    \begin{verbatim}

Call:
lm(formula = time ~ dist + climb + dist:climb, data = nihills.clean)

Residuals:
      Min        1Q    Median        3Q       Max 
-0.085241 -0.030444 -0.008553  0.039843  0.075714 

Coefficients:
              Estimate Std. Error t value Pr(>|t|)   
(Intercept) -5.662e-02  1.187e-01  -0.477  0.63970   
dist         8.866e-02  2.413e-02   3.674  0.00205 **
climb        1.599e-04  7.103e-05   2.252  0.03874 * 
dist:climb  -3.016e-07  1.317e-05  -0.023  0.98201   
---
Signif. codes:  0 ‘***’ 0.001 ‘**’ 0.01 ‘*’ 0.05 ‘.’ 0.1 ‘ ’ 1

Residual standard error: 0.04663 on 16 degrees of freedom
Multiple R-squared:  0.9462,	Adjusted R-squared:  0.9361 
F-statistic: 93.77 on 3 and 16 DF,  p-value: 2.292e-10

    \end{verbatim}

    
    \begin{Verbatim}[commandchars=\\\{\}]
{\color{incolor}In [{\color{incolor}84}]:} par\PY{p}{(}mfrow\PY{o}{=}\PY{k+kt}{c}\PY{p}{(}\PY{l+m}{2}\PY{p}{,}\PY{l+m}{2}\PY{p}{)}\PY{p}{)}
         plot\PY{p}{(}nihills2.clean.lm\PY{p}{)}
\end{Verbatim}


    \begin{center}
    \adjustimage{max size={0.9\linewidth}{0.9\paperheight}}{output_17_0.png}
    \end{center}
    { \hspace*{\fill} \\}
    
    \begin{Verbatim}[commandchars=\\\{\}]
{\color{incolor}In [{\color{incolor}87}]:} anova\PY{p}{(}nihills.clean.lm\PY{p}{,} nihills2.clean.lm\PY{p}{)}
\end{Verbatim}


    \begin{tabular}{r|llllll}
 Res.Df & RSS & Df & Sum of Sq & F & Pr(>F)\\
\hline
	 17           & 0.03479176   & NA           &           NA &          NA  &        NA   \\
	 16           & 0.03479062   &  1           & 1.140574e-06 & 0.000524543  & 0.9820109   \\
\end{tabular}


    
    \subsection{conclusions:}\label{conclusions}

Removing just 3 data points, the F-ratio decreases to a non-significant
value, with a non-significant p-value. Removing high-leverage points
revealed the low level of correlation between the variables, as seen in
the pairplot.

This high leverage could be solved by taking the logarithm of the
variables, thus reducing the extreme leverage and Cook's distance that
can be observed in the QQ-plot and residual plot of the first linear
model.

As extra work, the log transform is applied here, showing an improvement
even without removing outsiders:

    \begin{Verbatim}[commandchars=\\\{\}]
{\color{incolor}In [{\color{incolor}92}]:} pairs.panels\PY{p}{(}\PY{k+kp}{log}\PY{p}{(}nihills\PY{p}{)}\PY{p}{,} gap\PY{o}{=}\PY{l+m}{0}\PY{p}{)}
\end{Verbatim}


    \begin{center}
    \adjustimage{max size={0.9\linewidth}{0.9\paperheight}}{output_20_0.png}
    \end{center}
    { \hspace*{\fill} \\}
    
    \begin{Verbatim}[commandchars=\\\{\}]
{\color{incolor}In [{\color{incolor}88}]:} nihills.log.lm \PY{o}{\PYZlt{}\PYZhy{}} lm\PY{p}{(}\PY{k+kp}{log}\PY{p}{(}time\PY{p}{)} \PY{o}{\PYZti{}} \PY{k+kp}{log}\PY{p}{(}dist\PY{p}{)}\PY{o}{+}\PY{k+kp}{log}\PY{p}{(}climb\PY{p}{)}           \PY{p}{,} data\PY{o}{=}nihills.clean\PY{p}{)}
\end{Verbatim}


    \begin{Verbatim}[commandchars=\\\{\}]
{\color{incolor}In [{\color{incolor}89}]:} \PY{k+kp}{summary}\PY{p}{(}nihills.log.lm\PY{p}{)}
\end{Verbatim}


    
    \begin{verbatim}

Call:
lm(formula = log(time) ~ log(dist) + log(climb), data = nihills.clean)

Residuals:
      Min        1Q    Median        3Q       Max 
-0.173486 -0.047388 -0.009307  0.065425  0.113188 

Coefficients:
            Estimate Std. Error t value Pr(>|t|)    
(Intercept) -4.71238    0.34848 -13.523 1.59e-10 ***
log(dist)    0.64529    0.06924   9.319 4.31e-08 ***
log(climb)   0.43877    0.05100   8.603 1.34e-07 ***
---
Signif. codes:  0 ‘***’ 0.001 ‘**’ 0.01 ‘*’ 0.05 ‘.’ 0.1 ‘ ’ 1

Residual standard error: 0.07733 on 17 degrees of freedom
Multiple R-squared:  0.9368,	Adjusted R-squared:  0.9293 
F-statistic: 125.9 on 2 and 17 DF,  p-value: 6.426e-11

    \end{verbatim}

    
    \begin{Verbatim}[commandchars=\\\{\}]
{\color{incolor}In [{\color{incolor}90}]:} par\PY{p}{(}mfrow\PY{o}{=}\PY{k+kt}{c}\PY{p}{(}\PY{l+m}{2}\PY{p}{,}\PY{l+m}{2}\PY{p}{)}\PY{p}{)}
         plot\PY{p}{(}nihills.log.lm\PY{p}{)}
\end{Verbatim}


    \begin{center}
    \adjustimage{max size={0.9\linewidth}{0.9\paperheight}}{output_23_0.png}
    \end{center}
    { \hspace*{\fill} \\}
    

    % Add a bibliography block to the postdoc
    
    
    
    \end{document}
