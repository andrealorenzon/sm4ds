
% Default to the notebook output style

    


% Inherit from the specified cell style.




    
\documentclass[11pt]{article}

    
    
    \usepackage[T1]{fontenc}
    % Nicer default font (+ math font) than Computer Modern for most use cases
    \usepackage{mathpazo}

    % Basic figure setup, for now with no caption control since it's done
    % automatically by Pandoc (which extracts ![](path) syntax from Markdown).
    \usepackage{graphicx}
    % We will generate all images so they have a width \maxwidth. This means
    % that they will get their normal width if they fit onto the page, but
    % are scaled down if they would overflow the margins.
    \makeatletter
    \def\maxwidth{\ifdim\Gin@nat@width>\linewidth\linewidth
    \else\Gin@nat@width\fi}
    \makeatother
    \let\Oldincludegraphics\includegraphics
    % Set max figure width to be 80% of text width, for now hardcoded.
    \renewcommand{\includegraphics}[1]{\Oldincludegraphics[width=.8\maxwidth]{#1}}
    % Ensure that by default, figures have no caption (until we provide a
    % proper Figure object with a Caption API and a way to capture that
    % in the conversion process - todo).
    \usepackage{caption}
    \DeclareCaptionLabelFormat{nolabel}{}
    \captionsetup{labelformat=nolabel}

    \usepackage{adjustbox} % Used to constrain images to a maximum size 
    \usepackage{xcolor} % Allow colors to be defined
    \usepackage{enumerate} % Needed for markdown enumerations to work
    \usepackage{geometry} % Used to adjust the document margins
    \usepackage{amsmath} % Equations
    \usepackage{amssymb} % Equations
    \usepackage{textcomp} % defines textquotesingle
    % Hack from http://tex.stackexchange.com/a/47451/13684:
    \AtBeginDocument{%
        \def\PYZsq{\textquotesingle}% Upright quotes in Pygmentized code
    }
    \usepackage{upquote} % Upright quotes for verbatim code
    \usepackage{eurosym} % defines \euro
    \usepackage[mathletters]{ucs} % Extended unicode (utf-8) support
    \usepackage[utf8x]{inputenc} % Allow utf-8 characters in the tex document
    \usepackage{fancyvrb} % verbatim replacement that allows latex
    \usepackage{grffile} % extends the file name processing of package graphics 
                         % to support a larger range 
    % The hyperref package gives us a pdf with properly built
    % internal navigation ('pdf bookmarks' for the table of contents,
    % internal cross-reference links, web links for URLs, etc.)
    \usepackage{hyperref}
    \usepackage{longtable} % longtable support required by pandoc >1.10
    \usepackage{booktabs}  % table support for pandoc > 1.12.2
    \usepackage[inline]{enumitem} % IRkernel/repr support (it uses the enumerate* environment)
    \usepackage[normalem]{ulem} % ulem is needed to support strikethroughs (\sout)
                                % normalem makes italics be italics, not underlines
    

    
    
    % Colors for the hyperref package
    \definecolor{urlcolor}{rgb}{0,.145,.698}
    \definecolor{linkcolor}{rgb}{.71,0.21,0.01}
    \definecolor{citecolor}{rgb}{.12,.54,.11}

    % ANSI colors
    \definecolor{ansi-black}{HTML}{3E424D}
    \definecolor{ansi-black-intense}{HTML}{282C36}
    \definecolor{ansi-red}{HTML}{E75C58}
    \definecolor{ansi-red-intense}{HTML}{B22B31}
    \definecolor{ansi-green}{HTML}{00A250}
    \definecolor{ansi-green-intense}{HTML}{007427}
    \definecolor{ansi-yellow}{HTML}{DDB62B}
    \definecolor{ansi-yellow-intense}{HTML}{B27D12}
    \definecolor{ansi-blue}{HTML}{208FFB}
    \definecolor{ansi-blue-intense}{HTML}{0065CA}
    \definecolor{ansi-magenta}{HTML}{D160C4}
    \definecolor{ansi-magenta-intense}{HTML}{A03196}
    \definecolor{ansi-cyan}{HTML}{60C6C8}
    \definecolor{ansi-cyan-intense}{HTML}{258F8F}
    \definecolor{ansi-white}{HTML}{C5C1B4}
    \definecolor{ansi-white-intense}{HTML}{A1A6B2}

    % commands and environments needed by pandoc snippets
    % extracted from the output of `pandoc -s`
    \providecommand{\tightlist}{%
      \setlength{\itemsep}{0pt}\setlength{\parskip}{0pt}}
    \DefineVerbatimEnvironment{Highlighting}{Verbatim}{commandchars=\\\{\}}
    % Add ',fontsize=\small' for more characters per line
    \newenvironment{Shaded}{}{}
    \newcommand{\KeywordTok}[1]{\textcolor[rgb]{0.00,0.44,0.13}{\textbf{{#1}}}}
    \newcommand{\DataTypeTok}[1]{\textcolor[rgb]{0.56,0.13,0.00}{{#1}}}
    \newcommand{\DecValTok}[1]{\textcolor[rgb]{0.25,0.63,0.44}{{#1}}}
    \newcommand{\BaseNTok}[1]{\textcolor[rgb]{0.25,0.63,0.44}{{#1}}}
    \newcommand{\FloatTok}[1]{\textcolor[rgb]{0.25,0.63,0.44}{{#1}}}
    \newcommand{\CharTok}[1]{\textcolor[rgb]{0.25,0.44,0.63}{{#1}}}
    \newcommand{\StringTok}[1]{\textcolor[rgb]{0.25,0.44,0.63}{{#1}}}
    \newcommand{\CommentTok}[1]{\textcolor[rgb]{0.38,0.63,0.69}{\textit{{#1}}}}
    \newcommand{\OtherTok}[1]{\textcolor[rgb]{0.00,0.44,0.13}{{#1}}}
    \newcommand{\AlertTok}[1]{\textcolor[rgb]{1.00,0.00,0.00}{\textbf{{#1}}}}
    \newcommand{\FunctionTok}[1]{\textcolor[rgb]{0.02,0.16,0.49}{{#1}}}
    \newcommand{\RegionMarkerTok}[1]{{#1}}
    \newcommand{\ErrorTok}[1]{\textcolor[rgb]{1.00,0.00,0.00}{\textbf{{#1}}}}
    \newcommand{\NormalTok}[1]{{#1}}
    
    % Additional commands for more recent versions of Pandoc
    \newcommand{\ConstantTok}[1]{\textcolor[rgb]{0.53,0.00,0.00}{{#1}}}
    \newcommand{\SpecialCharTok}[1]{\textcolor[rgb]{0.25,0.44,0.63}{{#1}}}
    \newcommand{\VerbatimStringTok}[1]{\textcolor[rgb]{0.25,0.44,0.63}{{#1}}}
    \newcommand{\SpecialStringTok}[1]{\textcolor[rgb]{0.73,0.40,0.53}{{#1}}}
    \newcommand{\ImportTok}[1]{{#1}}
    \newcommand{\DocumentationTok}[1]{\textcolor[rgb]{0.73,0.13,0.13}{\textit{{#1}}}}
    \newcommand{\AnnotationTok}[1]{\textcolor[rgb]{0.38,0.63,0.69}{\textbf{\textit{{#1}}}}}
    \newcommand{\CommentVarTok}[1]{\textcolor[rgb]{0.38,0.63,0.69}{\textbf{\textit{{#1}}}}}
    \newcommand{\VariableTok}[1]{\textcolor[rgb]{0.10,0.09,0.49}{{#1}}}
    \newcommand{\ControlFlowTok}[1]{\textcolor[rgb]{0.00,0.44,0.13}{\textbf{{#1}}}}
    \newcommand{\OperatorTok}[1]{\textcolor[rgb]{0.40,0.40,0.40}{{#1}}}
    \newcommand{\BuiltInTok}[1]{{#1}}
    \newcommand{\ExtensionTok}[1]{{#1}}
    \newcommand{\PreprocessorTok}[1]{\textcolor[rgb]{0.74,0.48,0.00}{{#1}}}
    \newcommand{\AttributeTok}[1]{\textcolor[rgb]{0.49,0.56,0.16}{{#1}}}
    \newcommand{\InformationTok}[1]{\textcolor[rgb]{0.38,0.63,0.69}{\textbf{\textit{{#1}}}}}
    \newcommand{\WarningTok}[1]{\textcolor[rgb]{0.38,0.63,0.69}{\textbf{\textit{{#1}}}}}
    
    
    % Define a nice break command that doesn't care if a line doesn't already
    % exist.
    \def\br{\hspace*{\fill} \\* }
    % Math Jax compatability definitions
    \def\gt{>}
    \def\lt{<}
    % Document parameters
    \title{Assignment 5}
    
    
    

    % Pygments definitions
    
\makeatletter
\def\PY@reset{\let\PY@it=\relax \let\PY@bf=\relax%
    \let\PY@ul=\relax \let\PY@tc=\relax%
    \let\PY@bc=\relax \let\PY@ff=\relax}
\def\PY@tok#1{\csname PY@tok@#1\endcsname}
\def\PY@toks#1+{\ifx\relax#1\empty\else%
    \PY@tok{#1}\expandafter\PY@toks\fi}
\def\PY@do#1{\PY@bc{\PY@tc{\PY@ul{%
    \PY@it{\PY@bf{\PY@ff{#1}}}}}}}
\def\PY#1#2{\PY@reset\PY@toks#1+\relax+\PY@do{#2}}

\expandafter\def\csname PY@tok@w\endcsname{\def\PY@tc##1{\textcolor[rgb]{0.73,0.73,0.73}{##1}}}
\expandafter\def\csname PY@tok@c\endcsname{\let\PY@it=\textit\def\PY@tc##1{\textcolor[rgb]{0.25,0.50,0.50}{##1}}}
\expandafter\def\csname PY@tok@cp\endcsname{\def\PY@tc##1{\textcolor[rgb]{0.74,0.48,0.00}{##1}}}
\expandafter\def\csname PY@tok@k\endcsname{\let\PY@bf=\textbf\def\PY@tc##1{\textcolor[rgb]{0.00,0.50,0.00}{##1}}}
\expandafter\def\csname PY@tok@kp\endcsname{\def\PY@tc##1{\textcolor[rgb]{0.00,0.50,0.00}{##1}}}
\expandafter\def\csname PY@tok@kt\endcsname{\def\PY@tc##1{\textcolor[rgb]{0.69,0.00,0.25}{##1}}}
\expandafter\def\csname PY@tok@o\endcsname{\def\PY@tc##1{\textcolor[rgb]{0.40,0.40,0.40}{##1}}}
\expandafter\def\csname PY@tok@ow\endcsname{\let\PY@bf=\textbf\def\PY@tc##1{\textcolor[rgb]{0.67,0.13,1.00}{##1}}}
\expandafter\def\csname PY@tok@nb\endcsname{\def\PY@tc##1{\textcolor[rgb]{0.00,0.50,0.00}{##1}}}
\expandafter\def\csname PY@tok@nf\endcsname{\def\PY@tc##1{\textcolor[rgb]{0.00,0.00,1.00}{##1}}}
\expandafter\def\csname PY@tok@nc\endcsname{\let\PY@bf=\textbf\def\PY@tc##1{\textcolor[rgb]{0.00,0.00,1.00}{##1}}}
\expandafter\def\csname PY@tok@nn\endcsname{\let\PY@bf=\textbf\def\PY@tc##1{\textcolor[rgb]{0.00,0.00,1.00}{##1}}}
\expandafter\def\csname PY@tok@ne\endcsname{\let\PY@bf=\textbf\def\PY@tc##1{\textcolor[rgb]{0.82,0.25,0.23}{##1}}}
\expandafter\def\csname PY@tok@nv\endcsname{\def\PY@tc##1{\textcolor[rgb]{0.10,0.09,0.49}{##1}}}
\expandafter\def\csname PY@tok@no\endcsname{\def\PY@tc##1{\textcolor[rgb]{0.53,0.00,0.00}{##1}}}
\expandafter\def\csname PY@tok@nl\endcsname{\def\PY@tc##1{\textcolor[rgb]{0.63,0.63,0.00}{##1}}}
\expandafter\def\csname PY@tok@ni\endcsname{\let\PY@bf=\textbf\def\PY@tc##1{\textcolor[rgb]{0.60,0.60,0.60}{##1}}}
\expandafter\def\csname PY@tok@na\endcsname{\def\PY@tc##1{\textcolor[rgb]{0.49,0.56,0.16}{##1}}}
\expandafter\def\csname PY@tok@nt\endcsname{\let\PY@bf=\textbf\def\PY@tc##1{\textcolor[rgb]{0.00,0.50,0.00}{##1}}}
\expandafter\def\csname PY@tok@nd\endcsname{\def\PY@tc##1{\textcolor[rgb]{0.67,0.13,1.00}{##1}}}
\expandafter\def\csname PY@tok@s\endcsname{\def\PY@tc##1{\textcolor[rgb]{0.73,0.13,0.13}{##1}}}
\expandafter\def\csname PY@tok@sd\endcsname{\let\PY@it=\textit\def\PY@tc##1{\textcolor[rgb]{0.73,0.13,0.13}{##1}}}
\expandafter\def\csname PY@tok@si\endcsname{\let\PY@bf=\textbf\def\PY@tc##1{\textcolor[rgb]{0.73,0.40,0.53}{##1}}}
\expandafter\def\csname PY@tok@se\endcsname{\let\PY@bf=\textbf\def\PY@tc##1{\textcolor[rgb]{0.73,0.40,0.13}{##1}}}
\expandafter\def\csname PY@tok@sr\endcsname{\def\PY@tc##1{\textcolor[rgb]{0.73,0.40,0.53}{##1}}}
\expandafter\def\csname PY@tok@ss\endcsname{\def\PY@tc##1{\textcolor[rgb]{0.10,0.09,0.49}{##1}}}
\expandafter\def\csname PY@tok@sx\endcsname{\def\PY@tc##1{\textcolor[rgb]{0.00,0.50,0.00}{##1}}}
\expandafter\def\csname PY@tok@m\endcsname{\def\PY@tc##1{\textcolor[rgb]{0.40,0.40,0.40}{##1}}}
\expandafter\def\csname PY@tok@gh\endcsname{\let\PY@bf=\textbf\def\PY@tc##1{\textcolor[rgb]{0.00,0.00,0.50}{##1}}}
\expandafter\def\csname PY@tok@gu\endcsname{\let\PY@bf=\textbf\def\PY@tc##1{\textcolor[rgb]{0.50,0.00,0.50}{##1}}}
\expandafter\def\csname PY@tok@gd\endcsname{\def\PY@tc##1{\textcolor[rgb]{0.63,0.00,0.00}{##1}}}
\expandafter\def\csname PY@tok@gi\endcsname{\def\PY@tc##1{\textcolor[rgb]{0.00,0.63,0.00}{##1}}}
\expandafter\def\csname PY@tok@gr\endcsname{\def\PY@tc##1{\textcolor[rgb]{1.00,0.00,0.00}{##1}}}
\expandafter\def\csname PY@tok@ge\endcsname{\let\PY@it=\textit}
\expandafter\def\csname PY@tok@gs\endcsname{\let\PY@bf=\textbf}
\expandafter\def\csname PY@tok@gp\endcsname{\let\PY@bf=\textbf\def\PY@tc##1{\textcolor[rgb]{0.00,0.00,0.50}{##1}}}
\expandafter\def\csname PY@tok@go\endcsname{\def\PY@tc##1{\textcolor[rgb]{0.53,0.53,0.53}{##1}}}
\expandafter\def\csname PY@tok@gt\endcsname{\def\PY@tc##1{\textcolor[rgb]{0.00,0.27,0.87}{##1}}}
\expandafter\def\csname PY@tok@err\endcsname{\def\PY@bc##1{\setlength{\fboxsep}{0pt}\fcolorbox[rgb]{1.00,0.00,0.00}{1,1,1}{\strut ##1}}}
\expandafter\def\csname PY@tok@kc\endcsname{\let\PY@bf=\textbf\def\PY@tc##1{\textcolor[rgb]{0.00,0.50,0.00}{##1}}}
\expandafter\def\csname PY@tok@kd\endcsname{\let\PY@bf=\textbf\def\PY@tc##1{\textcolor[rgb]{0.00,0.50,0.00}{##1}}}
\expandafter\def\csname PY@tok@kn\endcsname{\let\PY@bf=\textbf\def\PY@tc##1{\textcolor[rgb]{0.00,0.50,0.00}{##1}}}
\expandafter\def\csname PY@tok@kr\endcsname{\let\PY@bf=\textbf\def\PY@tc##1{\textcolor[rgb]{0.00,0.50,0.00}{##1}}}
\expandafter\def\csname PY@tok@bp\endcsname{\def\PY@tc##1{\textcolor[rgb]{0.00,0.50,0.00}{##1}}}
\expandafter\def\csname PY@tok@fm\endcsname{\def\PY@tc##1{\textcolor[rgb]{0.00,0.00,1.00}{##1}}}
\expandafter\def\csname PY@tok@vc\endcsname{\def\PY@tc##1{\textcolor[rgb]{0.10,0.09,0.49}{##1}}}
\expandafter\def\csname PY@tok@vg\endcsname{\def\PY@tc##1{\textcolor[rgb]{0.10,0.09,0.49}{##1}}}
\expandafter\def\csname PY@tok@vi\endcsname{\def\PY@tc##1{\textcolor[rgb]{0.10,0.09,0.49}{##1}}}
\expandafter\def\csname PY@tok@vm\endcsname{\def\PY@tc##1{\textcolor[rgb]{0.10,0.09,0.49}{##1}}}
\expandafter\def\csname PY@tok@sa\endcsname{\def\PY@tc##1{\textcolor[rgb]{0.73,0.13,0.13}{##1}}}
\expandafter\def\csname PY@tok@sb\endcsname{\def\PY@tc##1{\textcolor[rgb]{0.73,0.13,0.13}{##1}}}
\expandafter\def\csname PY@tok@sc\endcsname{\def\PY@tc##1{\textcolor[rgb]{0.73,0.13,0.13}{##1}}}
\expandafter\def\csname PY@tok@dl\endcsname{\def\PY@tc##1{\textcolor[rgb]{0.73,0.13,0.13}{##1}}}
\expandafter\def\csname PY@tok@s2\endcsname{\def\PY@tc##1{\textcolor[rgb]{0.73,0.13,0.13}{##1}}}
\expandafter\def\csname PY@tok@sh\endcsname{\def\PY@tc##1{\textcolor[rgb]{0.73,0.13,0.13}{##1}}}
\expandafter\def\csname PY@tok@s1\endcsname{\def\PY@tc##1{\textcolor[rgb]{0.73,0.13,0.13}{##1}}}
\expandafter\def\csname PY@tok@mb\endcsname{\def\PY@tc##1{\textcolor[rgb]{0.40,0.40,0.40}{##1}}}
\expandafter\def\csname PY@tok@mf\endcsname{\def\PY@tc##1{\textcolor[rgb]{0.40,0.40,0.40}{##1}}}
\expandafter\def\csname PY@tok@mh\endcsname{\def\PY@tc##1{\textcolor[rgb]{0.40,0.40,0.40}{##1}}}
\expandafter\def\csname PY@tok@mi\endcsname{\def\PY@tc##1{\textcolor[rgb]{0.40,0.40,0.40}{##1}}}
\expandafter\def\csname PY@tok@il\endcsname{\def\PY@tc##1{\textcolor[rgb]{0.40,0.40,0.40}{##1}}}
\expandafter\def\csname PY@tok@mo\endcsname{\def\PY@tc##1{\textcolor[rgb]{0.40,0.40,0.40}{##1}}}
\expandafter\def\csname PY@tok@ch\endcsname{\let\PY@it=\textit\def\PY@tc##1{\textcolor[rgb]{0.25,0.50,0.50}{##1}}}
\expandafter\def\csname PY@tok@cm\endcsname{\let\PY@it=\textit\def\PY@tc##1{\textcolor[rgb]{0.25,0.50,0.50}{##1}}}
\expandafter\def\csname PY@tok@cpf\endcsname{\let\PY@it=\textit\def\PY@tc##1{\textcolor[rgb]{0.25,0.50,0.50}{##1}}}
\expandafter\def\csname PY@tok@c1\endcsname{\let\PY@it=\textit\def\PY@tc##1{\textcolor[rgb]{0.25,0.50,0.50}{##1}}}
\expandafter\def\csname PY@tok@cs\endcsname{\let\PY@it=\textit\def\PY@tc##1{\textcolor[rgb]{0.25,0.50,0.50}{##1}}}

\def\PYZbs{\char`\\}
\def\PYZus{\char`\_}
\def\PYZob{\char`\{}
\def\PYZcb{\char`\}}
\def\PYZca{\char`\^}
\def\PYZam{\char`\&}
\def\PYZlt{\char`\<}
\def\PYZgt{\char`\>}
\def\PYZsh{\char`\#}
\def\PYZpc{\char`\%}
\def\PYZdl{\char`\$}
\def\PYZhy{\char`\-}
\def\PYZsq{\char`\'}
\def\PYZdq{\char`\"}
\def\PYZti{\char`\~}
% for compatibility with earlier versions
\def\PYZat{@}
\def\PYZlb{[}
\def\PYZrb{]}
\makeatother


    % Exact colors from NB
    \definecolor{incolor}{rgb}{0.0, 0.0, 0.5}
    \definecolor{outcolor}{rgb}{0.545, 0.0, 0.0}



    
    % Prevent overflowing lines due to hard-to-break entities
    \sloppy 
    % Setup hyperref package
    \hypersetup{
      breaklinks=true,  % so long urls are correctly broken across lines
      colorlinks=true,
      urlcolor=urlcolor,
      linkcolor=linkcolor,
      citecolor=citecolor,
      }
    % Slightly bigger margins than the latex defaults
    
    \geometry{verbose,tmargin=1in,bmargin=1in,lmargin=1in,rmargin=1in}
    
    

    \begin{document}
    
    
    \maketitle
    
    

    
    Lorenzon\\
CS 1.3\\
CS 2.4\\
CS 4.3\\
BC 2.5\\
DAAG 7.5

    \hypertarget{cs-1.3}{%
\section{CS 1.3}\label{cs-1.3}}

    Suppose that \[ Y \sim 
\left(
\begin{bmatrix}
    1  \\
   2 
\end{bmatrix},
\begin{bmatrix}
2 & 1 \\
1 & 2  
\end{bmatrix}
\right) \]

Find the conditional p.d.f. of \(Y_1\) given that \(Y_1 + Y_2 = 3\)

    \textbf{Solution:}

The conditional (on \(Y_2 = 3 - Y_1\) ) distribution is a multivariate
normal \((Y_1 | Y_2 = 3 - Y_1) \sim N(\bar\mu , \bar\Sigma)\) where:

\(\bar\mu = \mu_1 + \Sigma_{12} \Sigma_{22}^{-1}(3 -Y_1 - \mu_2)\)

and covariance matrix:

\$ \bar\Sigma = \Sigma\emph{\{11\} -
\Sigma}\{12\}\Sigma\emph{\{22\}\^{}\{-1\}\Sigma}\{21\}\$

As a general case,
\[X_1 | X_2 = x_2 \sim Normal \left(\mu_1 + \frac{\sigma_1}{\sigma_2} \rho (x_2-\mu_2), (1-\rho^2)\sigma_1^2\right)\]

with \(\rho = -1\)

so

\[X_1 | X_2 = x_2 \sim Normal \left(\mu_1 - \frac{\sigma_1}{\sigma_2}  (x_2-\mu_2), (1-1)\sigma_1^2\right)\]

\[X_1 | X_2 = x_2 \sim Normal \left(\mu_1 - \frac{\sigma_1}{\sigma_2}  (x_2-\mu_2), 0\right)\]

From \(\Sigma\), \(\sigma_1 = \sigma_2\), so it reduces to:

\[X_1 | X_2 = x_2 \sim Normal \left(\mu_1 - (x_2-\mu_2), 0\right)\]

\[X_1 | X_2 = x_2 \sim Normal \left(\mu_1+\mu_2 - x_2, 0\right)\]

and from the problem

\[X_1 | X_2 = x_2 \sim Normal \left(3-x_2, 0\right)\]

It makes sense, since, if you know one of the variables, and the sum is
fixed, you know the other.

    \hypertarget{cs-2.4}{%
\section{CS 2.4}\label{cs-2.4}}

Using R, write a function to evaluate the log-likelihood of \(\theta_t\)
for example 4 in section 2.1. (Hint: see \texttt{?dexp}). Plot log
likelihood against \(\theta_t\) over a suitable range and making use of
(2.4) and the definition of a confidence interval, find a 95\%
confidence interval for \(\theta_t\) (\texttt{pchisq} is also useful).

    MLE for exponential distributions is \(\hat\lambda=\frac{1}{\bar x}\)

    \begin{Verbatim}[commandchars=\\\{\}]
{\color{incolor}In [{\color{incolor}1}]:} \PY{c+c1}{\PYZsh{}range of theta}
        theta \PY{o}{\PYZlt{}\PYZhy{}} seq \PY{p}{(}\PY{l+m}{0.00000001}\PY{p}{,} \PY{l+m}{0.01}\PY{p}{,} \PY{l+m}{0.00001}\PY{p}{)}
        
        \PY{c+c1}{\PYZsh{}data from book}
        xx \PY{o}{=} \PY{k+kt}{c} \PY{p}{(}\PY{l+m}{28}\PY{p}{,} \PY{l+m}{32}\PY{p}{,} \PY{l+m}{49}\PY{p}{,} \PY{l+m}{84}\PY{p}{,} \PY{l+m}{357}\PY{p}{,}\PY{l+m}{933}\PY{p}{,}\PY{l+m}{1078}\PY{p}{,}\PY{l+m}{1183}\PY{p}{,}\PY{l+m}{1560}\PY{p}{,}\PY{l+m}{2114}\PY{p}{,}\PY{l+m}{2144}\PY{p}{)}
        
        \PY{k+kp}{cat}\PY{p}{(}\PY{l+s}{\PYZdq{}}\PY{l+s}{mean: \PYZdq{}} \PY{p}{,} \PY{k+kp}{mean}\PY{p}{(}xx\PY{p}{)}\PY{p}{,} \PY{l+s}{\PYZdq{}}\PY{l+s}{\PYZbs{}n\PYZdq{}}\PY{p}{)} 
        \PY{k+kp}{cat}\PY{p}{(}\PY{l+s}{\PYZdq{}}\PY{l+s}{MLE : \PYZdq{}} \PY{p}{,} \PY{l+m}{1}\PY{o}{/}\PY{k+kp}{mean}\PY{p}{(}xx\PY{p}{)}\PY{p}{)} \PY{c+c1}{\PYZsh{}MLE}
        
        n\PY{o}{=} \PY{k+kp}{length}\PY{p}{(}xx\PY{p}{)}
        
        \PY{c+c1}{\PYZsh{}log likelihod}
        LLexp \PY{o}{\PYZlt{}\PYZhy{}} \PY{k+kr}{function} \PY{p}{(}theta\PY{p}{)} \PY{p}{\PYZob{}}
             \PY{k+kp}{log}\PY{p}{(}theta\PY{p}{)} \PY{o}{*} n \PY{o}{\PYZhy{}}theta\PY{o}{*}\PY{k+kp}{sum}\PY{p}{(}xx\PY{p}{)}
        \PY{p}{\PYZcb{}}
        
        \PY{c+c1}{\PYZsh{}likelihood, to compare peaks}
        Lexp \PY{o}{\PYZlt{}\PYZhy{}} \PY{k+kr}{function} \PY{p}{(}theta\PY{p}{)} \PY{p}{\PYZob{}}
            theta\PY{o}{\PYZca{}}n \PY{o}{*} \PY{k+kp}{exp}\PY{p}{(}\PY{o}{\PYZhy{}}theta\PY{o}{*}\PY{k+kp}{sum}\PY{p}{(}xx\PY{p}{)}\PY{p}{)}
        \PY{p}{\PYZcb{}}
\end{Verbatim}


    \begin{Verbatim}[commandchars=\\\{\}]
mean:  869.2727 
MLE :  0.001150387
    \end{Verbatim}

    \begin{Verbatim}[commandchars=\\\{\}]
{\color{incolor}In [{\color{incolor}17}]:} \PY{k+kp}{options}\PY{p}{(}warn\PY{o}{=}\PY{l+m}{\PYZhy{}1}\PY{p}{)} \PY{c+c1}{\PYZsh{}to temporarily suppress warnings about Nelder\PYZhy{}Mead optimization}
         
         \PY{c+c1}{\PYZsh{}plot loglikelihood with CI}
         negLLexp \PY{o}{\PYZlt{}\PYZhy{}} \PY{k+kr}{function} \PY{p}{(}\PY{k+kp}{t}\PY{p}{)} \PY{p}{\PYZob{}} \PY{o}{\PYZhy{}} LLexp\PY{p}{(}\PY{k+kp}{t}\PY{p}{)}\PY{p}{\PYZcb{}} \PY{c+c1}{\PYZsh{}negative log lik target function}
         opt \PY{o}{\PYZlt{}\PYZhy{}} optim\PY{p}{(}\PY{l+m}{1}\PY{p}{,}negLLexp\PY{p}{)}
         \PY{c+c1}{\PYZsh{}despite the warnings, the results are reliable}
         
         plot\PY{p}{(}theta\PY{p}{,}\PY{k+kp}{sapply}\PY{p}{(}theta\PY{p}{,} LLexp\PY{p}{)}\PY{p}{,} type\PY{o}{=}\PY{l+s}{\PYZdq{}}\PY{l+s}{l\PYZdq{}}\PY{p}{,} xlim\PY{o}{=}\PY{k+kt}{c}\PY{p}{(}\PY{l+m}{0}\PY{p}{,}\PY{l+m}{0.003}\PY{p}{)}\PY{p}{,} ylim\PY{o}{=}\PY{k+kt}{c}\PY{p}{(}\PY{l+m}{\PYZhy{}100}\PY{p}{,}\PY{l+m}{\PYZhy{}80}\PY{p}{)}\PY{p}{,} ylab \PY{o}{=} \PY{l+s}{\PYZdq{}}\PY{l+s}{log\PYZhy{}likelihood\PYZdq{}}\PY{p}{)}
         abline\PY{p}{(}v\PY{o}{=}\PY{l+m}{1}\PY{o}{/}\PY{k+kp}{mean}\PY{p}{(}xx\PY{p}{)}\PY{p}{,} lty\PY{o}{=}\PY{l+m}{2}\PY{p}{,} col\PY{o}{=}\PY{l+s}{\PYZdq{}}\PY{l+s}{green\PYZdq{}}\PY{p}{)}  \PY{c+c1}{\PYZsh{}MLE}
         abline\PY{p}{(}v\PY{o}{=}opt\PY{o}{\PYZdl{}}par\PY{p}{,} lty\PY{o}{=}\PY{l+m}{3}\PY{p}{,} col\PY{o}{=}\PY{l+s}{\PYZdq{}}\PY{l+s}{blue\PYZdq{}}\PY{p}{)} \PY{c+c1}{\PYZsh{}perfect superposition with MLE}
         abline\PY{p}{(}h\PY{o}{=}\PY{o}{\PYZhy{}}opt\PY{o}{\PYZdl{}}value\PY{p}{,} lty\PY{o}{=}\PY{l+m}{2}\PY{p}{)} \PY{c+c1}{\PYZsh{}max L}
         abline\PY{p}{(}h\PY{o}{=}\PY{o}{\PYZhy{}}opt\PY{o}{\PYZdl{}}value\PY{o}{\PYZhy{}}qchisq\PY{p}{(}\PY{l+m}{.95}\PY{p}{,}\PY{l+m}{1}\PY{p}{)}\PY{o}{/}\PY{l+m}{2}\PY{p}{,} lty\PY{o}{=}\PY{l+m}{1}\PY{p}{,} col\PY{o}{=}\PY{l+s}{\PYZdq{}}\PY{l+s}{red\PYZdq{}}\PY{p}{)}  \PY{c+c1}{\PYZsh{}95\PYZpc{} CI for ML}
         
         reasonable\PYZus{}theta\PYZus{}values \PY{o}{\PYZlt{}\PYZhy{}} theta\PY{p}{[}LLexp\PY{p}{(}theta\PY{p}{)}\PY{o}{\PYZgt{}=}\PY{p}{(}\PY{o}{\PYZhy{}}opt\PY{o}{\PYZdl{}}value\PY{o}{\PYZhy{}}qchisq\PY{p}{(}\PY{l+m}{.95}\PY{p}{,}\PY{l+m}{1}\PY{p}{)}\PY{o}{/}\PY{l+m}{2}\PY{p}{)}\PY{p}{]}
         
         \PY{c+c1}{\PYZsh{}in blue, 95\PYZpc{} CI for ML thetas}
         abline\PY{p}{(}v\PY{o}{=}\PY{k+kp}{min}\PY{p}{(}reasonable\PYZus{}theta\PYZus{}values\PY{p}{)}\PY{p}{,} col\PY{o}{=}\PY{l+s}{\PYZdq{}}\PY{l+s}{blue\PYZdq{}}\PY{p}{)}
         abline\PY{p}{(}v\PY{o}{=}\PY{k+kp}{max}\PY{p}{(}reasonable\PYZus{}theta\PYZus{}values\PY{p}{)}\PY{p}{,} col\PY{o}{=}\PY{l+s}{\PYZdq{}}\PY{l+s}{blue\PYZdq{}}\PY{p}{)}
\end{Verbatim}


    \begin{center}
    \adjustimage{max size={0.9\linewidth}{0.9\paperheight}}{output_7_0.png}
    \end{center}
    { \hspace*{\fill} \\}
    
    \begin{Verbatim}[commandchars=\\\{\}]
{\color{incolor}In [{\color{incolor}13}]:} cat \PY{p}{(}\PY{l+s}{\PYZdq{}}\PY{l+s}{95\PYZpc{} CI for theta: \PYZdq{}}\PY{p}{)}
         \PY{k+kp}{cat}\PY{p}{(}\PY{k+kp}{min}\PY{p}{(}reasonable\PYZus{}theta\PYZus{}values\PY{p}{)}\PY{p}{)}
         \PY{k+kp}{cat}\PY{p}{(}\PY{l+s}{\PYZdq{}}\PY{l+s}{ \PYZhy{} \PYZdq{}}\PY{p}{)}
         \PY{k+kp}{cat}\PY{p}{(}\PY{k+kp}{max}\PY{p}{(}reasonable\PYZus{}theta\PYZus{}values\PY{p}{)}\PY{p}{)}
\end{Verbatim}


    \begin{Verbatim}[commandchars=\\\{\}]
95\% CI for theta: 0.00060001 - 0.00197001
    \end{Verbatim}

    \hypertarget{cs-4.3}{%
\section{CS 4.3}\label{cs-4.3}}

Random variables \(X\) and \(Y\) have a joint p.d.f.
\(f(x,y)= k x^\alpha y^\beta , 0 \le x\le 1, 0 \le y\le 1\).

Assume that you have \(n\) independent pairs of observations
\((x_i, y_i)\):

\begin{enumerate}
\def\labelenumi{\alph{enumi})}
\tightlist
\item
  evaluate \(k\) in terms of \(\alpha\) and \(\beta\).
\end{enumerate}

    \[ k \int_0^1 \int_o^1 x^\alpha y^\beta dx dy =  \frac{k}{(\alpha+1)(\beta+1)}=1\]

so

\[ k = (\alpha+1)(\beta+1)\]

    \begin{enumerate}
\def\labelenumi{\alph{enumi})}
\setcounter{enumi}{1}
\tightlist
\item
  find the ML estimators of \(\alpha\) and \(\beta\)
\end{enumerate}

    \[MLE(\theta) = n (log(\alpha+1) + log(\beta+1)) + \sum{\alpha log(x_i)} + \sum{\beta log(y_i)}\]

\[ \frac{\partial l}{\partial \alpha} = \frac{n}{\alpha+1} + \sum{log(x_i)}\]
and similarly
\[ \frac{\partial l}{\partial \beta} = \frac{n}{\beta+1} + \sum{log(y_i)}\]

that gives:

\[\hat\alpha = \frac{-n}{\sum log(x_i)}-1 \hat\beta = \frac{-n}{\sum log(y_i)}-1 \]

    \begin{enumerate}
\def\labelenumi{\alph{enumi})}
\setcounter{enumi}{2}
\tightlist
\item
  find the appropriate variances of \(\hat\alpha\) and \(\hat\beta\)
\end{enumerate}

    \[\frac{\partial^2l}{\partial\beta^2}=\frac{-n}{(\beta+1)^2}\]
\[\frac{\partial^2l}{\partial\alpha^2}=\frac{-n}{(\alpha+1)^2}\]
\[\frac{\partial^2l}{\partial\alpha\partial\beta}=0\]

hence:

\[var(\hat\alpha)\approx\frac{(\hat\alpha+1)^2}{n}\]

\[var(\hat\beta)\approx\frac{(\hat\beta+1)^2}{n}\]

    \hypertarget{bc-2.5}{%
\section{BC 2.5}\label{bc-2.5}}

\textbf{Estimate a normal mean with discrete prior.}

Suppose you are interested in estimating the total average snowfall per
year \(\mu\) (in inches) for a large city on the East Coast of the
United States. Assume individual yearly snow totals \(y_1, ... , y_n\)
are collected from a population that is assumed to be normally
distributed with mean \(\mu\) and known standard deviation \(\sigma\) =
10 inches.

\begin{enumerate}
\def\labelenumi{\alph{enumi})}
\item
  before collecting data, suppose you believe that the mean snowfall
  \(\mu\) can be the values 20,30,40,50,60,70 inches with the following
  probabilities:

  mu \textbar{} 20 30 40 50 60 70 prior \textbar{} .1 .15 .25 .25 .15 .1
\end{enumerate}

Place the \(\mu\) probabilities in the vector \texttt{mu} and the
associated prior probabilities in the vector \texttt{prior}.

    \begin{Verbatim}[commandchars=\\\{\}]
{\color{incolor}In [{\color{incolor}238}]:} mu \PY{o}{\PYZlt{}\PYZhy{}} \PY{k+kt}{c}\PY{p}{(}\PY{l+m}{20}\PY{p}{,}\PY{l+m}{30}\PY{p}{,}\PY{l+m}{40}\PY{p}{,}\PY{l+m}{50}\PY{p}{,}\PY{l+m}{60}\PY{p}{,}\PY{l+m}{70}\PY{p}{)}
          prior \PY{o}{\PYZlt{}\PYZhy{}} \PY{k+kt}{c}\PY{p}{(}\PY{l+m}{.1}\PY{p}{,}\PY{l+m}{.15}\PY{p}{,}\PY{l+m}{.25}\PY{p}{,}\PY{l+m}{.25}\PY{p}{,}\PY{l+m}{.15}\PY{p}{,}\PY{l+m}{.1}\PY{p}{)}
          sigma \PY{o}{\PYZlt{}\PYZhy{}} \PY{l+m}{10}
          plot\PY{p}{(}mu\PY{p}{,}prior\PY{p}{,} type\PY{o}{=}\PY{l+s}{\PYZdq{}}\PY{l+s}{l\PYZdq{}}\PY{p}{,} ylim\PY{o}{=}\PY{k+kt}{c}\PY{p}{(}\PY{l+m}{0}\PY{p}{,}\PY{l+m}{0.3}\PY{p}{)}\PY{p}{)}
\end{Verbatim}


    \begin{center}
    \adjustimage{max size={0.9\linewidth}{0.9\paperheight}}{output_16_0.png}
    \end{center}
    { \hspace*{\fill} \\}
    
    \begin{enumerate}
\def\labelenumi{\alph{enumi})}
\setcounter{enumi}{1}
\tightlist
\item
  Suppose you observe the yearly snowfall totals 38.6, 42.4, 57.5, 40.5,
  51.7, 67.1, 33.4, 60.9, 64.1, 40.1, 40.7. Enter these data in the
  vector \texttt{y} and compute the sample mean \texttt{ybar}.
\end{enumerate}

    \begin{Verbatim}[commandchars=\\\{\}]
{\color{incolor}In [{\color{incolor}239}]:} y \PY{o}{\PYZlt{}\PYZhy{}} \PY{k+kt}{c}\PY{p}{(}\PY{l+m}{38.6}\PY{p}{,} \PY{l+m}{42.4}\PY{p}{,} \PY{l+m}{57.5}\PY{p}{,} \PY{l+m}{40.5}\PY{p}{,} \PY{l+m}{51.7}\PY{p}{,} \PY{l+m}{67.1}\PY{p}{,} \PY{l+m}{33.4}\PY{p}{,} \PY{l+m}{60.9}\PY{p}{,} \PY{l+m}{64.1}\PY{p}{,} \PY{l+m}{40.1}\PY{p}{,} \PY{l+m}{40.7}\PY{p}{)}
          ybar \PY{o}{\PYZlt{}\PYZhy{}} \PY{k+kp}{mean}\PY{p}{(}y\PY{p}{)}
          n \PY{o}{\PYZlt{}\PYZhy{}} \PY{k+kp}{length}\PY{p}{(}y\PY{p}{)}
          \PY{k+kp}{cat}\PY{p}{(}\PY{l+s}{\PYZdq{}}\PY{l+s}{y\PYZus{}bar = \PYZdq{}}\PY{p}{,} ybar\PY{p}{)}
\end{Verbatim}


    \begin{Verbatim}[commandchars=\\\{\}]
y\_bar =  48.81818
    \end{Verbatim}

    \begin{enumerate}
\def\labelenumi{\alph{enumi})}
\setcounter{enumi}{2}
\tightlist
\item
  In this problem, the likelihood function is given by
  \[ L(\mu) = exp (-\frac{n}{2\sigma}(\mu-\bar y)^2)\] where \(\bar y\)
  is the sample mean. Compute the likelihood of the list of values in
  \texttt{mu} and place the likelihood values in the vector
  \texttt{like}.
\end{enumerate}

    \begin{Verbatim}[commandchars=\\\{\}]
{\color{incolor}In [{\color{incolor}254}]:} norm\PYZus{}likelihood \PY{o}{\PYZlt{}\PYZhy{}} \PY{k+kr}{function} \PY{p}{(}m\PY{p}{)} \PY{p}{\PYZob{}}
              exp \PY{p}{(}\PY{o}{\PYZhy{}} n\PY{o}{/}\PY{p}{(}\PY{l+m}{2}\PY{o}{*}sigma\PY{p}{)} \PY{o}{*} \PY{p}{(}m \PY{o}{\PYZhy{}} ybar\PY{p}{)}\PY{o}{\PYZca{}}\PY{l+m}{2}\PY{p}{)}
          \PY{p}{\PYZcb{}}
          
          like \PY{o}{\PYZlt{}\PYZhy{}} \PY{k+kp}{Vectorize}\PY{p}{(}norm\PYZus{}likelihood\PY{p}{)}\PY{p}{(}mu\PY{p}{)}
          like
          
          plot\PY{p}{(}mu\PY{p}{,}\PY{k+kp}{log}\PY{p}{(}like\PY{p}{)}\PY{p}{,} type\PY{o}{=}\PY{l+s}{\PYZdq{}}\PY{l+s}{l\PYZdq{}}\PY{p}{)}
\end{Verbatim}


    \begin{enumerate*}
\item 4.2471649409776e-199
\item 2.58972343022305e-85
\item 2.66694853674476e-19
\item 0.463855676028506
\item 1.36256775074928e-30
\item 6.75990086613985e-108
\end{enumerate*}


    
    \begin{center}
    \adjustimage{max size={0.9\linewidth}{0.9\paperheight}}{output_20_1.png}
    \end{center}
    { \hspace*{\fill} \\}
    
    \begin{enumerate}
\def\labelenumi{\alph{enumi})}
\setcounter{enumi}{3}
\tightlist
\item
  one can compute the posterior probability for \(\mu\) using the
  formula \texttt{post=prior*like/sum(prior*like)}. Compute the
  posterior probabilities of \(\mu\) for this example.
\end{enumerate}

    \begin{Verbatim}[commandchars=\\\{\}]
{\color{incolor}In [{\color{incolor}256}]:} post \PY{o}{=} prior \PY{o}{*} like \PY{o}{/} \PY{k+kp}{sum}\PY{p}{(}prior\PY{o}{*}like\PY{p}{)}
          plot\PY{p}{(}mu\PY{p}{,}post\PY{p}{,} type\PY{o}{=}\PY{l+s}{\PYZdq{}}\PY{l+s}{l\PYZdq{}}\PY{p}{)}
\end{Verbatim}


    \begin{center}
    \adjustimage{max size={0.9\linewidth}{0.9\paperheight}}{output_22_0.png}
    \end{center}
    { \hspace*{\fill} \\}
    
    \begin{enumerate}
\def\labelenumi{\alph{enumi})}
\setcounter{enumi}{4}
\tightlist
\item
  Using the function \texttt{discint}, find an 80\% probability interval
  for \(\mu\).
\end{enumerate}

    \begin{Verbatim}[commandchars=\\\{\}]
{\color{incolor}In [{\color{incolor}260}]:} \PY{k+kn}{library}\PY{p}{(}LearnBayes\PY{p}{)}
          dist \PY{o}{=} \PY{k+kp}{cbind}\PY{p}{(}mu\PY{p}{,}post\PY{p}{)}
          pcontent \PY{o}{=} \PY{l+m}{.8}
          discint\PY{p}{(}dist\PY{p}{,}pcontent\PY{p}{)}
          
          \PY{c+c1}{\PYZsh{} well, being just few discrete probability, that was expected somehow, and deluding.}
          \PY{c+c1}{\PYZsh{} only 50 has a high probability of being the posterior.}
          \PY{c+c1}{\PYZsh{} we could fit a distribution on our posterior, compute }
          \PY{c+c1}{\PYZsh{} a continuous distribution, and consider .8 Prob. Interval on that, but it is not asked.}
\end{Verbatim}


    \begin{description}
\item[\$prob] 1
\item[\$set] 50
\end{description}


    
    \hypertarget{daag-7.5}{%
\section{DAAG 7.5}\label{daag-7.5}}

The data frame \texttt{cuckoos} holds data on the lengths and breadths
of eggs of cuckoos, found in the nests of six different species of host
birds. Fit models for the regression of length on breadth that have:

A: a single line for all six species. B: different parallel lines for
the different host species. C: separate lines for the separate host
species.

Use the \texttt{anova()} function to print out the sequential analysis
of variance table. Which of the three models is preferred? Print out the
diagnostic plots for this model. Do they show anything worthy of note?
Examine the output coefficients from this model carefully, and decide
whether the results seem grouped by host species. How might the results
be summarized for reporting purposes?

    \begin{Verbatim}[commandchars=\\\{\}]
{\color{incolor}In [{\color{incolor}206}]:} \PY{c+c1}{\PYZsh{}a lot of libraries, because I experimented too.}
          
          \PY{k+kn}{library}\PY{p}{(}DAAG\PY{p}{)}
          \PY{k+kn}{library}\PY{p}{(}lme4\PY{p}{)}  \PY{c+c1}{\PYZsh{}lmer}
          \PY{k+kn}{library}\PY{p}{(}dplyr\PY{p}{)}
          \PY{k+kn}{library}\PY{p}{(}ggplot2\PY{p}{)}
          \PY{k+kn}{library}\PY{p}{(}MASS\PY{p}{)}
          \PY{k+kn}{library}\PY{p}{(}broom\PY{p}{)}
          \PY{k+kn}{library}\PY{p}{(}purrr\PY{p}{)}
          \PY{k+kn}{library}\PY{p}{(}tibble\PY{p}{)}
          \PY{k+kn}{library}\PY{p}{(}caret\PY{p}{)}
          \PY{k+kn}{library}\PY{p}{(}mltools\PY{p}{)}
          \PY{k+kn}{library}\PY{p}{(}tidyr\PY{p}{)}
          \PY{k+kn}{library}\PY{p}{(}tidyverse\PY{p}{)}
          \PY{c+c1}{\PYZsh{}attach(cuckoos)}
\end{Verbatim}


    \begin{Verbatim}[commandchars=\\\{\}]
{\color{incolor}In [{\color{incolor}217}]:} \PY{c+c1}{\PYZsh{} A: a single line for all six species}
          \PY{c+c1}{\PYZsh{} for this problem a simple geom\PYZus{}smooth will suffice. Bonus: confidence region}
          
          model1 \PY{o}{=} lm \PY{p}{(}length \PY{o}{\PYZti{}} breadth\PY{p}{,} data \PY{o}{=} cuckoos\PY{p}{)}
          
          ggplot\PY{p}{(}data \PY{o}{=} cuckoos\PY{p}{,} aes\PY{p}{(}x \PY{o}{=} breadth\PY{p}{,} y \PY{o}{=}\PY{k+kp}{length}\PY{p}{)}\PY{p}{)} \PY{o}{+}
              geom\PYZus{}point\PY{p}{(}alpha \PY{o}{=} \PY{l+m}{0.8}\PY{p}{,} aes\PY{p}{(}color \PY{o}{=} species\PY{p}{)}\PY{p}{)} \PY{o}{+}
              stat\PYZus{}smooth\PY{p}{(}method\PY{o}{=} \PY{l+s}{\PYZdq{}}\PY{l+s}{lm\PYZdq{}}\PY{p}{,} fill\PY{o}{=}\PY{l+s}{\PYZdq{}}\PY{l+s}{blue\PYZdq{}}\PY{p}{,}  size\PY{o}{=}\PY{l+m}{2}\PY{p}{,} alpha\PY{o}{=}\PY{l+m}{0.1}\PY{p}{)}
\end{Verbatim}


    \begin{center}
    \adjustimage{max size={0.9\linewidth}{0.9\paperheight}}{output_27_0.png}
    \end{center}
    { \hspace*{\fill} \\}
    
    \begin{Verbatim}[commandchars=\\\{\}]
{\color{incolor}In [{\color{incolor}223}]:} \PY{c+c1}{\PYZsh{}B : different parallel lines for different host species}
          
          model4 \PY{o}{\PYZlt{}\PYZhy{}} lm\PY{p}{(}length \PY{o}{\PYZti{}} breadth \PY{o}{+} species\PY{p}{,} data \PY{o}{=} cuckoos\PY{p}{)}
          augmented\PYZus{}data \PY{o}{=} augment\PY{p}{(}model4\PY{p}{)}
          \PY{c+c1}{\PYZsh{}glimpse(augmented\PYZus{}data)}
          
          data\PYZus{}space \PY{o}{\PYZlt{}\PYZhy{}} ggplot\PY{p}{(}augmented\PYZus{}data\PY{p}{,} aes\PY{p}{(}x \PY{o}{=} breadth\PY{p}{,} y \PY{o}{=} \PY{k+kp}{length}\PY{p}{,} color\PY{o}{=}species\PY{p}{)}\PY{p}{)} \PY{o}{+} 
                    geom\PYZus{}point\PY{p}{(}\PY{p}{)}
          
          data\PYZus{}space \PY{o}{+} 
                geom\PYZus{}line\PY{p}{(}aes\PY{p}{(}y \PY{o}{=} \PY{l+m}{.}fitted\PY{p}{)}\PY{p}{)}
\end{Verbatim}


    \begin{center}
    \adjustimage{max size={0.9\linewidth}{0.9\paperheight}}{output_28_0.png}
    \end{center}
    { \hspace*{\fill} \\}
    
    \begin{Verbatim}[commandchars=\\\{\}]
{\color{incolor}In [{\color{incolor}224}]:} \PY{c+c1}{\PYZsh{} C: separate line for the separate host species}
          
          \PY{c+c1}{\PYZsh{}get all species}
          species\PYZus{}labels \PY{o}{=} \PY{k+kp}{unique}\PY{p}{(}cuckoos\PY{o}{\PYZdl{}}species\PY{p}{)}
          
          models \PY{o}{=} \PY{k+kt}{list}\PY{p}{(}\PY{p}{)}
          
          \PY{k+kr}{for} \PY{p}{(}specie \PY{k+kr}{in} species\PYZus{}labels\PY{p}{)}\PY{p}{\PYZob{}}
              data  \PY{o}{\PYZlt{}\PYZhy{}} cuckoos \PY{o}{\PYZpc{}\PYZgt{}\PYZpc{}} filter\PY{p}{(}species\PY{o}{==}specie\PY{p}{)}
              model \PY{o}{\PYZlt{}\PYZhy{}} lm \PY{p}{(}length \PY{o}{\PYZti{}} breadth\PY{p}{,} data \PY{o}{=} data\PY{p}{)}
              models\PY{p}{[}specie\PY{p}{]} \PY{o}{=} model
              \PY{p}{\PYZcb{}}
\end{Verbatim}


    \begin{Verbatim}[commandchars=\\\{\}]
{\color{incolor}In [{\color{incolor}237}]:} ggplot\PY{p}{(}data \PY{o}{=} cuckoos\PY{p}{,} aes\PY{p}{(}x \PY{o}{=} breadth\PY{p}{,} y \PY{o}{=}\PY{k+kp}{length}\PY{p}{)}\PY{p}{)} \PY{o}{+}
              geom\PYZus{}point\PY{p}{(}alpha \PY{o}{=} \PY{l+m}{0.8}\PY{p}{,} aes\PY{p}{(}color \PY{o}{=} species\PY{p}{)}\PY{p}{)} \PY{o}{+}
              geom\PYZus{}abline\PY{p}{(}slope \PY{o}{=} models\PY{o}{\PYZdl{}}meadow.pipit\PY{p}{[}\PY{l+m}{2}\PY{p}{]}\PY{p}{,} intercept \PY{o}{=} models\PY{o}{\PYZdl{}}meadow.pipit\PY{p}{[}\PY{l+m}{1}\PY{p}{]}\PY{p}{,} color\PY{o}{=}\PY{l+s}{\PYZdq{}}\PY{l+s}{darkgoldenrod\PYZdq{}}\PY{p}{)} \PY{o}{+}
              geom\PYZus{}abline\PY{p}{(}slope \PY{o}{=} models\PY{o}{\PYZdl{}}hedge.sparrow\PY{p}{[}\PY{l+m}{2}\PY{p}{]}\PY{p}{,} intercept \PY{o}{=} models\PY{o}{\PYZdl{}}hedge.sparrow\PY{p}{[}\PY{l+m}{1}\PY{p}{]}\PY{p}{,} color\PY{o}{=}\PY{l+s}{\PYZdq{}}\PY{l+s}{coral1\PYZdq{}}\PY{p}{)} \PY{o}{+}
              geom\PYZus{}abline\PY{p}{(}slope \PY{o}{=} models\PY{o}{\PYZdl{}}pied.wagtail\PY{p}{[}\PY{l+m}{2}\PY{p}{]}\PY{p}{,} intercept \PY{o}{=} models\PY{o}{\PYZdl{}}pied.wagtail\PY{p}{[}\PY{l+m}{1}\PY{p}{]}\PY{p}{,} color\PY{o}{=}\PY{l+s}{\PYZdq{}}\PY{l+s}{aquamarine3\PYZdq{}}\PY{p}{)} \PY{o}{+}
              geom\PYZus{}abline\PY{p}{(}slope \PY{o}{=} models\PY{o}{\PYZdl{}}robin\PY{p}{[}\PY{l+m}{2}\PY{p}{]}\PY{p}{,} intercept \PY{o}{=} models\PY{o}{\PYZdl{}}robin\PY{p}{[}\PY{l+m}{1}\PY{p}{]}\PY{p}{,} color\PY{o}{=}\PY{l+s}{\PYZdq{}}\PY{l+s}{cyan2\PYZdq{}}\PY{p}{)} \PY{o}{+}
              geom\PYZus{}abline\PY{p}{(}slope \PY{o}{=} models\PY{o}{\PYZdl{}}tree.pipit\PY{p}{[}\PY{l+m}{2}\PY{p}{]}\PY{p}{,} intercept \PY{o}{=} models\PY{o}{\PYZdl{}}tree.pipit\PY{p}{[}\PY{l+m}{1}\PY{p}{]}\PY{p}{,} color\PY{o}{=}\PY{l+s}{\PYZdq{}}\PY{l+s}{cornflowerblue\PYZdq{}}\PY{p}{)} \PY{o}{+}
              geom\PYZus{}abline\PY{p}{(}slope \PY{o}{=} models\PY{o}{\PYZdl{}}wren\PY{p}{[}\PY{l+m}{2}\PY{p}{]}\PY{p}{,} intercept \PY{o}{=} models\PY{o}{\PYZdl{}}wren\PY{p}{[}\PY{l+m}{1}\PY{p}{]}\PY{p}{,} color\PY{o}{=}\PY{l+s}{\PYZdq{}}\PY{l+s}{hotpink3\PYZdq{}}\PY{p}{)}
\end{Verbatim}


    \begin{center}
    \adjustimage{max size={0.9\linewidth}{0.9\paperheight}}{output_30_0.png}
    \end{center}
    { \hspace*{\fill} \\}
    

    % Add a bibliography block to the postdoc
    
    
    
    \end{document}
